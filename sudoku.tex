\input{preamble}
\input{format}
\input{commands}

\begin{document}

\begin{Large}
    \textsf{\textbf{Sudoku mit SMT}}

    Formale Methoden und Werkzeuge WS24/25 - Oscar Friske
\end{Large}
\vspace{2ex}

Dieses Projekt implementiert einen Sudoku Solver, sowie einen Generator für neue Sudoku Instanzen. Zur Modellierung des SMT Problems wurde die Bibliothek "hasmtlib" sowie Haskell als Hostsprache genutzt

\vspace{2ex}

\begin{Large}
    SMT Spezifikationen
\end{Large}
\begin{itemize}
  \item Solver: Z3
  \item Logik: QF\_LIA - linear integer arithmetic
\end{itemize}
\vspace{2ex}

\begin{Large}
    Implementierung
\end{Large}
\vspace{2ex}

Das Projekt ist grundsätzlich in 2 Teile unterteilt:
\begin{enumerate}[(1)]
    \item Solve.hs \newline
    Der Solver zum Lösen eines Sudokus verwendet verwendet und modelliert ein SMT Problem mit 3 wesentlichen Constraints
        \begin{enumerate}[(a)]
            \item Row Constraint \newline
            Jede Zeile muss genau die Zahlen 1-9 enthalten (keine Doppelungen)
            \item Column Constraint \newline
            Jede Spalte muss genau die Zahlen 1-9 enthalten (keine Doppelungen)
            \item Subgrid Constraint \newline
            Jedes 3x3 Feld muss genau die Zahlen 1-9 enthalten (keine Doppelungen)
        \end{enumerate}
    Desweiteren gibt es zwei Constraint, welchen den Zahlenraum einschränken (1-9), ein Constraint, welches die vordefinierten Felder voraussetzt und ein optionales Constraint, welches eine Lösung ausschließt (später wichtig für die Aufgaben Instanziierung)
    \item Generate.hs \newline
    Auf Basis des Solvers können neue Instanzen eines Sudokus generiert werden. Die grundsätzliche Idee ist dabei folgende:
        \begin{enumerate}[(1)]
            \item Generiere \(n\) zufällige Felder (in meinem Beispiel 10) und Löse basierend auf diesen ein Sudokubrett
            \item Entferne ein zufälliges Feld (die Lösbarkeit bleibt dabei erhalten)
            \item Überprüfe, ob das Sudoku weiterhin nur eine mögliche Lösung hat. \newline
            Falls ja, gehe zu Schritt (2) \newline
            Falls nein, mache den letzten Schritt rückgängig und lösche ein anderes Feld. Wurde nach \(x\) Schritten (in meinem Beispiel 15) keine eindeutige Instanz mit weniger vorgegebenen Feldern gefunde, beende die Generierung und geben den aktuellen Stand zurück. Falls doch, gehe zu Schritt (2)
        \end{enumerate}
\end{enumerate}
Um zu überprüfen, ob das aktuelle Brett nur eine mögliche Lösung hat, wird nach
\bibliographystyle{apalike}

\end{document}
